\documentclass[11pt]{article}
\usepackage[utf8]{inputenc}
\usepackage[margin=0.3in]{geometry}
\usepackage{hyperref}
%---------------------------------------
\title{Basic Time-Series}
\author{Pramod Duvvuri}
\date{April 20, 2020}
\begin{document}
	\maketitle
	The below notes are written to accompany the book \textit{Forecasting: Principles and Practice} by \textbf{Rob J Hyndman and George Athanasopoulo}, one should be familiar with the basics of statistical concepts. One should also be familiar with Regression. The language of choice is \textbf{R} and the IDE is RStudio and we shall be using the \textit{ggplot2} package of the \textit{tidyverse} to plot, analyze and draw conclusions from the data we have.
	\begin{enumerate}
	\item Mains goals of forecasting
	\item Types of forecasting models
	\begin{enumerate}
		\item Explanatory models (Prediction using features)
		\item Forecasting models (Prediction using Lag values)
		\item Mixed models (Combination of the above two models)
	\end{enumerate}
     \item Basic Steps in a forecasting task
     \item Time-Series Graphics
     \begin{enumerate}
     	\item Time-plots
     	\item Seasonal Plots \& Polar Seasonal Plots
     	\item Seasonal Sub-series Plots
     	\item Scatter Plots
     	\item Lag Plots
     	\item ACF/Correlogram Plots
     \end{enumerate}
      \item Components of a Time-series
      \begin{enumerate}
      	\item Trend
      	\item Seasonal
      	\item Cyclic
      	\item Error
      \end{enumerate}
       \item White noise (ACF Plot to confirm)
       \item Simple Forecasting Techniques
       \begin{enumerate}
       	\item Average method
       	\item Naive method
       	\item Seasonal Naive method
       	\item Drift method
       \end{enumerate}
	\end{enumerate}
\end{document}	