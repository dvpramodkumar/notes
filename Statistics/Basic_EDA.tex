\documentclass[11pt]{article}
\usepackage[utf8]{inputenc}
\usepackage[margin=0.3in]{geometry}
\usepackage{hyperref}
%---------------------------------------
\title{Basic EDA}
\author{Pramod Duvvuri}
\date{March 31, 2019}
\begin{document}
	\maketitle
	The below notes are written to accompany the book \textit{Visualizing Data} by William Cleveland and the S670 class notes written by Prof. Dr. Brad Luen. Before learning Exploratory Data Analysis (EDA), one should be familiar with the basics of statistical concepts. One should also be familiar with Regression. The language of choice is \textbf{R} and the IDE is RStudio and we shall be using the \textit{ggplot2} package of the \textit{tidyverse} to plot, analyze and draw conclusions from the data we have.
	\begin{enumerate}
		\item Differences between CDA/EDA
		\item What is EDA ?
		\begin{enumerate}
			\item Graphing
			\item Fitting
		\end{enumerate}
	    \item The need for EDA
		\item Univariate Data (singer data in lattice package)
		\begin{enumerate}
			\item Histogram
			\item Density Plot
			\item Boxplot
			\item ECDF
			\item Normal QQ Plot
			\item Tukey Mean difference Plot
			\item Additive Shift
			\item Fitting a linear model
		\end{enumerate}
	\item Bivariate Data
	\begin{enumerate}
		\item Scatter Plot
	\end{enumerate}
	\end{enumerate}
\subsection*{ggplot2 functions}
\begin{verbatim}
 ECDF - stat_ecdf()
 Histogram - geom_histogram()
 Density Plot - geom_density()
 Boxplot - geom_boxplot()
 Quantile Plot - stat_qq()
 Facet Grid - facet_grid()
 Facet Wrap - facet_wrap() \\ m x n display
 Scatter Plot - geom_point()
 Line - geom_abline()
 QQ Plot - qqplot() \\ Base R function
 Flip Axes - coord_flip()
\end{verbatim}
\end{document}	